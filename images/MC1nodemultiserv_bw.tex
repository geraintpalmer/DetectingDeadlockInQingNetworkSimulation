\documentclass{standalone}

\usepackage{tikz}
\usetikzlibrary{arrows}
\usetikzlibrary{decorations.markings}
\usetikzlibrary{calc}

\begin{document}

\begin{tikzpicture}
    \tikzstyle{state}=[minimum width=1.5cm, font=\boldmath];
    % First col
    \node (0) at (0,0) [state] {\Large $(0)$};
    \node (1) at ($(0)+(3.5,0)$) [state] {\Large $(1)$};
    \node (2) at ($(1)+(3.5,0)$) [state] {\Large $(2)$};
    \node (dots1) at ($(2)+(3.5,0)$) [state] {\Huge...};
    \node (c-1) at ($(dots1)+(3.5,0)$) [state] {\Large $(c-1)$};
    \node (c) at ($(c-1)+(3.5,0)$) [state] {\Large $(c)$};
    \node (c+1) at ($(c)+(0,-3.5)$) [state] {\Large $(c+1)$};
    \node (dots2) at ($(c+1)+(0,-3.5)$) [state] {\Huge...};
    \node (n+c-1) at ($(dots2)+(0,-3.5)$) [state] {\Large $(n+c-1)$};

    % Second col
    \node (n+c) at ($(n+c-1)+(-3.5,0)$) [state] {\Large $(n+c)$};
    \node (n+c+1) at ($(n+c)+(-3.5,0)$) [state] {\Large $(n+c+1)$};
    \node (dots3) at ($(n+c+1)+(-3.5,0)$) [state] {\Huge...};
    \node (n+2c-1) at ($(dots3)+(-3.5,0)$) [state] {\Large $(n+2c-1)$};
    \node (n+2c) at ($(n+2c-1)+(-3.5,0)$) [state] {\Large $(n+2c)$};

    % % Transitions
    \draw (0) edge[out=45,in=135,->,thick] node [above] {\footnotesize$\Lambda$} (1);
    \draw (1) edge[out=45,in=135,->,thick] node [above] {\footnotesize$\Lambda$} (2);
    \draw (2) edge[out=45,in=180,->,thick] node [above, pos=1.0] {\footnotesize$\Lambda$} ($(2)+(1.5,0.9)$);
    \draw ($(c-1)+(-1.5,0.9)$) edge[out=0,in=135,->,thick] node [above, pos=0.0] {\footnotesize$\Lambda$} (c-1);
    \draw (c-1) edge[out=45,in=135,->,thick] node [above] {\footnotesize$\Lambda$} (c);

    \draw (c) edge[out=-45,in=45,->,thick] node [right] {\footnotesize$\Lambda$} (c+1);
    \draw (c+1) edge[out=-45,in=90,->,thick] node [right] {\footnotesize$\Lambda$} ($(c+1)+(0.8,-1.75)$);
    \draw ($(n+c-1)+(0.8,1.75)$) edge[out=-90,in=45,->,thick] node [right] {\footnotesize$\Lambda$} (n+c-1);

    \draw (n+c-1) edge[out=135,in=45,->,thick] node [below] {\footnotesize$cr_{11}\mu$} (n+c);
    \draw (n+c) edge[out=135,in=45,->,thick] node [above] {\footnotesize$cr_{11}\mu$} (n+c+1);
    \draw (n+c+1) edge[out=135,in=0,->,thick] node [above, pos=1.0] {\footnotesize$(c-1)r_{11}\mu$} ($(n+c+1)+(-1.5,0.9)$);
    \draw ($(n+2c-1)+(1.5,0.9)$) edge[out=180,in=45,->,thick] node [above, pos=0.0] {\footnotesize$2r_{11}\mu$} (n+2c-1);
    \draw (n+2c-1) edge[out=135,in=45,->,thick] node [above] {\footnotesize$r_{11}\mu$} (n+2c);

    \draw (0) edge[out=-45,in=-135,<-,thick] node [below] {\footnotesize$(1-r_{11})\mu$} (1);
    \draw (1) edge[out=-45,in=-135,<-,thick] node [below] {\footnotesize$2(1-r_{11})\mu$} (2);
    \draw (2) edge[out=-45,in=180,<-,thick] node [below, pos=1.0] {\footnotesize$3(1-r_{11})\mu$} ($(2)+(1.5,-0.9)$);
    \draw ($(c-1)+(-1.5,-0.9)$) edge[out=0,in=-135,->,thick] node [below, pos=0.0] {\footnotesize$(c-1)(1-r_{11})\mu$} (c-1);
    \draw (c-1) edge[out=-45,in=-135,<-,thick] node [above] {\footnotesize$c(1-r_{11})\mu$} (c);


    \draw (c) edge[out=-135,in=135,<-,thick] node [below, rotate=90] {\footnotesize$c(1-r_{11})\mu$} (c+1);
    \draw (c+1) edge[out=-135,in=90,<-,thick] node [below left, rotate=90] {\footnotesize$c(1-r_{11})\mu$} ($(c+1)+(-0.8,-1.75)$);
    \draw ($(n+c-1)+(-0.8,1.75)$) edge[out=-90,in=135,<-,thick] node [below right, rotate=90] {\footnotesize$c(1-r_{11})\mu$} (n+c-1);

    \draw (n+c) edge[out=-45,in=-135,->,thick] node [left, pos=0.35] {\footnotesize$c(1-r_{11})\mu$} (n+c-1);
    \draw (n+c+1) edge[out=-40,in=-120,->,thick] node [below left, pos=0.25] {\footnotesize$(c-1)(1-r_{11})\mu$} (n+c-1);
    \draw (n+2c-1) edge[out=-35,in=-105,->,thick] node [below left, pos=0.25] {\footnotesize$(1-r_{11})\mu$} (n+c-1);


\end{tikzpicture}

\end{document}