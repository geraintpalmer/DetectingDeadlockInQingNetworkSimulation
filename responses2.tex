\documentclass{article}

\usepackage{fullpage}
\usepackage{parskip}
\usepackage{setspace}
\usepackage{amsmath}
\usepackage{amsfonts}
\usepackage{amsthm}

\title{Responses to Reviewers}
\author{}
\date{}


\begin{document}

\maketitle

We would like to express our gratitude to the editor and reviewers for their
useful suggestions and remarks on the manuscript.

The rest of this document will address specific concerns raised by the editor
and reviewers.

\section*{Responses to Reviewer 5}

\begin{itemize}

\item Reviewer 5 wrote:
\begin{quote}
``At the end of the Introduction, I would like to see a stronger and clearer statement of (what the authors see as) the main contributions of the paper. This should in essence answer the questions ‘Why is this work important? Why does it merit publication in EJOR?’''
\end{quote}

\item Reviewer 5 wrote:
\begin{quote}
``Now that (as it seems to me) Section 4 is the centrepiece of the work, the material here needs improvement. The proofs of Theorem 1 and Proposition 1 need some serious work to make them mathematically compelling.''
\end{quote}


\item Reviewer 5 wrote:
\begin{quote}
``Reading the proof of Theorem 1, it is not at all clear to me that you have proved the statement in the result. For a start, the statement is ‘if and only if’ and it is not at all clear to me that you have the implication going both ways. There is no mention of strongly connected components in the proof and yet that is a key property of a knot. I want to be able to see absolutely clearly that you can start from ‘Suppose a deadlock state arises at t’ and from there infer that ‘D(t) must contain a knot’ and then repeat with the implications reversed.''
\end{quote}


\item Reviewer 5 wrote:
\begin{quote}
``So far as Proposition 1 is concerned, I assume that (starting from Theorem 1) the claim is that in the three examples of queueing networks listed the statements ‘D(t) contains a knot’ and ‘D(t) contains a weakly connected component without a sink node’ are equivalent (ie, are ‘if and only if’). Is that what you have proved? Please make it much clearer. As with Theorem 1, in the proofs the implications appear to go only one way, which is a concern. One further point, please explain the value of Proposition 1. Is it easier to check that D(t) has a weakly component without a sink node than that it contains a knot? Do you use Proposition 1 in any of your own analyses?''
\end{quote}

\item Reviewer 5 wrote:
\begin{quote}
``You do not seem to have addressed my concerns about Section 6. I do not see any explanation given as to why what you have done here makes any useful contribution to the work. In Figure 14 you have already demonstrated an ability to compute the mean time to deadlock analytically for these systems. Presumably you can also estimate these quantities by simulation, using the ideas in Section 4. I imagine that you can also do the latter for rather larger and more complex networks than the ones discussed here. What value then is there in producing (what are often) poor bounds for these quantities? If the bounds were rather better, I could imagine that
they might have a role in system design to (for example) ameliorate deadlock by identifying embedded sub-systems which are vulnerable to early deadlock. However, anything of that kind seems a long way off.''
\end{quote}

and

\begin{quote}
``My own personal preference would be for a shorter paper which omits Section 6''
\end{quote}

Section 6 has been removed. We agree that although interesting, the content in its current form does not have as much usefulness as other sections.


\item Reviewer 5 wrote:
\begin{quote}
``p.1 Abstract: ‘These models are compared to results obtained .........’. I do not think this is quite what you mean;''
\end{quote}


\item Reviewer 5 wrote:
\begin{quote}
``p.1 Last four lines of the first para of Section 1. Please reword from ‘These deadlocks can be real......’ to achieve greater clarity;
''
\end{quote}


\item Reviewer 5 wrote:
\begin{quote}
``p.2 Definition 1 and material following. Please aim for greater clarity here. If you relate Figure 1 to Definition 1, what exactly is the ‘subset of customers’ in the latter which is relevant here? Is it just Asub1 and Bsub1?''
\end{quote}


\item Reviewer 5 wrote:
\begin{quote}
``p.3 bottom para to the end of Section 2. Please try to reword for clarity. This discussion could be clearer on the distinction between real-world deadlock and the model-based variety. It should also explain more clearly why the latter (which is the focus of the paper) matters;''
\end{quote}


\item Reviewer 5 wrote:
\begin{quote}
``pp4-6 The literature review is very long. If it is possible to lose some of the material toward the end of Section 3, that might help readability;''
\end{quote}


\item Reviewer 5 wrote:
\begin{quote}
``p.7 l.3 up In the definition of ‘weakly connected’ does the ‘either/or’ include ‘both and’? In other words, is a strongly connected vertex pair necessarily weakly connected?''
\end{quote}


\item Reviewer 5 wrote:
\begin{quote}
``p.10, l.2 up Should it not be ....’to check whether any strongly connected component....’''
\end{quote}


\item Reviewer 5 wrote:
\begin{quote}
``p.11, l.11 You claim ‘In this section......expressions for their expected time to deadlock are found.’ What are you referring to here? I do not see any ‘expressions’ unless you are claiming (5.1) meets this description.''
\end{quote}


\item Reviewer 5 wrote:
\begin{quote}
``p.11, l.3up In (5.1) you give a formula for the ‘expected number of time steps until absorption’ for a (discrete) Markov chain. Will you not need to uniformise your continuous chains to translate this easily into the ‘expected time to deadlock’? If so, should this be mentioned? How exactly did you compute the ‘analytical mean’ times to deadlock in Figure 8 etc?''
\end{quote}

\item Reviewer 5 wrote:
\begin{quote}
``p.15, l.9 ....’the rate at which i are (is?) reduced for most states’....Please explain. What does this mean?''
\end{quote}

\item Reviewer 5 wrote:
\begin{quote}
``p.15 Are Remarks 1 and 2 mathematical certainties or empirical observations? If the former, can they be proved rigorously?''
\end{quote}

\item Reviewer 5 wrote:
\begin{quote}
``p.16, l.3 I think ‘possibilities’ should be ‘probabilities’ here;''
\end{quote}


\item Reviewer 5 wrote:
\begin{quote}
``p.16, bottom In your description of state space S, the index applied to N should be 2 rather than the one given. The space S consists of pairs of integers. A similar comment applies on p.19. Also please rewrite the material following the characterisation of S, some of which seems confusing and contradictory;''
\end{quote}


\item Reviewer 5 wrote:
\begin{quote}
``p.18 (and later). Having commented extensively on the analytical/simulation results given in Figure 8, you say very little about the Figures 11 and 14.''
\end{quote}


\end{itemize}

\end{document}