\documentclass{article}

\usepackage{fullpage}
\usepackage{parskip}
\usepackage{setspace}
\usepackage{amsmath}
\usepackage{amsfonts}
\usepackage{amsthm}

\title{Responses to Reviewers}
\author{}
\date{}


\begin{document}

\maketitle

We would like to express our gratitude to the editor and reviewers for their
useful suggestions and remarks on the manuscript.
These recommendation resulted in a significant restructure of parts of the
paper, along with many other stylistic and content changes throughout.
These changes include:

\begin{itemize}
	\item Restructure of the introduction.
	\item Inclusion of a motivating real world example.
	\item Rewritten definition of deadlock.
	\item Inclusion of further references.
	\item Improvements to diagrams and plots.
	\item Removal of superfluous Markov chain models.
\end{itemize}

The rest of this document will address specific concerns raised by the editor
and reviewers.

\section*{Responses to the Editor}

\begin{itemize}

\item The editor wrote:
\begin{quote}
``Please ensure that you have cited recent and relevant publications in EJOR
and other OR journals, to avoid the impression that your article would be
misplaced in EJOR and to link your findings better to current OR trends
(cf. Report \#5).''
\end{quote}
Some recent publications, including some from EJOR, have now been cited:
\begin{itemize}
\item ``An analytic finite capacity queueing network model capturing the
propagation of congestion and blocking'', Osorio, C. and Bierlaire, M., 2009.
\item ``Application of a simulation-based dynamic traffic assignment model'',
Florian, M., Mahut, M. and Tremblay, N., 2008.
\item ``A sufficient condition for the liveness of weighted event graphs'',
Marchetti, O. and Munier-Kordon, A., 2009.
\item ``Modeling recirculating conveyors with blocking'', Schmidt, L. C and
Jackman, J., 2000.
\item ``Survey of research in the design and control of automated guided
vehicle systems'', Vis, I.F.A., 2006.
\item ``From bed-blocking to delayed discharges: precursors and
interpretations of a contested concept'', Manzano-Santaella, A., 2010.
\end{itemize}


\item The editor wrote:
\begin{quote}
``I also would suggest to include a line of acknowledgement to the referees
in your article.''
\end{quote}
This has been included as the final section.

\end{itemize}


\section*{Responses to Reviewer 1}

\begin{itemize}

\item Reviewer 1 wrote:
\begin{quote}
``The Markov model presentation is quite repetitive and it could be simplified
since the models are simple and the results can be easily related.
I suggest the author to reduce Section 3 to avoid useless repetition of the
models details and the observations on the model behaviour.
The model definition and analysis is quite standard technique and there is no
particular original contribution in this part of the work.''
\end{quote}
Of the original five networks modelled, only three are now considered.
The single server one node system was a special case of the multi server one
node system.
The single server two node system was a special case of the multi server two
node system.
The single server $\Omega$ system was kept, along with multi server $\Omega_1$
and $\Omega_2$ systems, as discussion on these models are needed for the
section on the bound on the expected time to deadlock.


\item Reviewer 1 wrote:
\begin{quote}
``The bound definition could be interesting, however this section of the work
is quite confused.
It is not clear how to apply the bounds.''
\end{quote}
This section has been expanded and rewritten.
A table (Table 2) of values showing how the bound performs is now included.
This should also help clarify the meaning of the bound.
Further work could build on the ideas presented in this section, to derive
bounds for larger networks that have not yet been (and maybe unable to be)
modelled analytically.
This note on further work is added to the conclusions section.


\item Reviewer 1 wrote:
\begin{quote}
``It seems that the bound is defined just for one of the considered models,
but it is not clear its applicability and accuracy''
\end{quote}
The bound is only defined for model $\Omega$, in terms of $\Omega_1$ and
$\Omega_2$.
However, a discussion of further work, to apply the ideas presented here to
networks that are not modelled explicitly, is included in the conclusions.
For the considered models, accuracy and tightness of the bound is analysed and
discussed in this section.

\item Reviewer 1 wrote:
\begin{quote}
``the figures are not readable for the state transition diagrams (Fig. 9, 12,
15, 18)''
\end{quote}

and:

\begin{quote}
``the graphs of the expected time to deadlock cannot be read (Fig. 7, 10, 13,
16, 17, 23)''
\end{quote}

The clarity of these diagrams and plots has been improved.
We appreciate the reviewer pointing this out.


\item Reviewer 1 wrote:
\begin{quote}
``the presentation of the bound definition of the models (p. 24) can be
simplified.''
\end{quote}
This section has been simplified.

\end{itemize}



\section*{Responses to Reviewer 2}

\begin{itemize}

\item Reviewer 2 wrote:
\begin{quote}
``There is no doubt that deadlock is a bad phenomenon in queuing networks.
However, it is difficult to understand the importance of deadlock detection by
applied researcher.
Therefore, the author(s) should explain the importance of deadlock detection
by using a concrete example.
The author(s) should identify problems that occur because of deadlock in
practice.''
\end{quote}

and:

\begin{quote}
``To ensure the paper is reader friendly, the author(s) should indicate which
operational problems can be solved by using this study’s result.''
\end{quote}

and:

\begin{quote}
``The case is not really made for the importance of blocking/deadlock in
practice for the systems discussed.
I have seen real-world instances and am personally convinced of the importance
of these phenomena, but the paper has to make the case.
In my view, the paper is crying out for a motivating real-world example;''
\end{quote}

A motivational example is given in an introductory section.
This example shows a potential relationship between community care and
secondary care, and reflects the situation modelled as an $\Omega_2$ system.
Furthermore, an article is referenced in which potential deadlock is described
as occurring in a real healthcare scenario, a Swiss hospital.
We now emphasise that deadlock is a phenomenon and problem of the model, not
of reality.
In practice customers are swapped, but current modelling of circular Type I
blocking does not account for this, and will deadlock.
This emphasises the potential application of the work in developing simulation
and analytical models that may exhibit deadlock despite the system itself not.


\item Reviewer 2 wrote:
\begin{quote}
``Why does this study consider only Type 1 deadlock?
The author(s) must indicate why this study does not consider deadlocks of
Types 2 and 3.''
\end{quote}

Although this paper does only consider blocking of Type I, consideration of
Types II and III blocking are now noted as future research directions.
It is noted that blocking of type III with random destinations (RS-RD) cannot
reach deadlock, although could be considered a method of deadlock prevention.

\item Reviewer 2 wrote:
\begin{quote}
``The author(s) should justify some assumptions.
For example, why does this study pay attention to no pre-emption?''
\end{quote}

In the literature review we state that there are four conditions that a system
must satisfy for deadlock to occur.
We then show that open restricted queueing network satisfy all four conditions.
One of the conditions is no-preemption, and in queueing networks this is
satisfies of there are either no priorities, or priorities with no-preemption.
This work focuses on systems without priorities.
This paragraph is clarified in the text.

\item Reviewer 2 wrote:
\begin{quote}
``This paper has some references missing.
For example, in [14], the journal name is mentioned as “Operations Research.”
However, in [17] and [30], the journal name is mentioned as “Operations
research”. The author(s) should check the format of the references.''
\end{quote}

This has been addressed, we appreciate the reviewer taking the time to point
these out.

\end{itemize}



\section*{Responses to Reviewer 5}

\begin{itemize}

\item Reviewer 5 wrote:
\begin{quote}
``Many of the terms used in the paper are not defined/explained with the
clarity I would wish;''
\end{quote}
\begin{itemize}
\item The introduction has been restructured, and now defines terms before
using them.
\item The definition of deadlock has been rewritten for clarity, and now does
not use any undefined terms.
\item A list of graph theoretical terms has been provided prior to discussion
on graph theory.
\end{itemize}


\item Reviewer 5 wrote:
\begin{quote}
``It is not made clear just how much of an advance the central result of
Section 2 (Theorem 1) is on previous work.
Is the construction in Definition 2 new, for example?''
\end{quote}

and:

\begin{quote}
``The contribution of Section 2 needs to be more clearly set in the context
of the foregoing literature.
The suggestion on p.4 is that rather similar approaches (using ‘wait-for
graphs’) have been used previously to analyse other systems.
Please identify more clearly what is the novel contribution here;''
\end{quote}

Yes the construction in Definition 2 is new.
This methods is a generic method for all FIFO queueing networks under Type I
blocking, whereas previous work seems bespoke to the system concerned.
New introductory sentences to this Section have been included to explain and
clarify this and to put the work in context.


\item Reviewer 5 wrote:
\begin{quote}
``I struggle to see the importance of much of the material in Sections 3 and 4.
I would have thought that after the general discussion preceding Section 3.1
a single illustrative example would suffice.
Further, while the bounding result in Section 4 seems correct, what is its
significance and where does it leave us?
Is it meant to point to a general approach to bounding expected times to
deadlock in more complex systems?
If this is the intention, the case does not seem to have been made.
The bound proposed does not appear particularly tight.
But more than that, no case is made regarding why and when such a bound
would be of value.''
\end{quote}
The bound isn't very tight, and there is a discussion on its tightness and
accuracy is this section.
We have also included a table (Table 2) of example values to illustrate this.
The ideas presented in the proof of the bound, however, could be used to
develop bounds for systems that have not been modelled analytically, by
considering the networks embedded within.
Notes on this further work have been included in the conclusions.

The number of examples in the modelling section has been reduced from 5 to 3.
Three remain as discussion on these three systems is required in order to
define the bound.

\item Reviewer 5 wrote:
\begin{quote}
``Abstract. It is not good to have a number of unfamiliar yet undefined terms
(eg, restricted queueing network, recursive upstream blocking, deadlock etc)
in an abstract.''
\end{quote}
Some unfamiliar terms have been replaced, although some have remained.
I feel that ``open restricted queueing network'', will be
familiar to readers interested in queueing networks.
As this is the central idea of the paper, I do not think that the abstract
could be written without the word ``deadlock''.
The abstract would not be an appropriate place to define this term either.

\item Reviewer 5 wrote:
\begin{quote}
``Further, the upper bound is on the expected time to deadlock.
The time to deadlock is a random variable;''
\end{quote}
The description of the bound has been updated to clarify that it is for the
expected time to deadlock.

\item Reviewer 5 wrote:
\begin{quote}
``Introduction: I found this very long and quite unclear in parts.
While it collects together some interesting material, I feel it lacks clarity
and direction.
I strongly suggest that you begin with a clear and precise description of the
phenomena of blocking/deadlock in the systems of interest to you.
Explain why they matter.
Then describe succinctly the contribution of the paper.
Proceed to a review of the literature (or perhaps this could be a separate
section?) and end with a summary of the contents of the paper;''
\end{quote}
The introduction has been restructured, following this suggestion.
Literature review is now a section of its own, a motivating example is
provided.
The structure of the introduction is now:
\begin{itemize}
\item Define ``open restricted queueing networks'', and explain Type I blocking
\item Define and explain deadlock.
\item Explain notation used throughout paper
\item Motivating example (separate section)
\item Literature review (separate section)
\end{itemize}

\item Reviewer 5 wrote:
\begin{quote}
``We really need to understand what blocking is and why you are taking the
view of it you are.
For example, in the one node single-server model of Section 3.1, why does the
proposed ‘-1’ state mean deadlock for the system?
In your deadlock state, why cannot the served customer to be rerouted to the
queue when the waiting space is full take up the very space she vacates when
she leaves service?
Presumably your notion of blocking forbids this, but please explain clearly
what your notions of blocking are and why they are natural/important for
applications;''
\end{quote}

The reviewer has touched on an excellent and important point.
The fact that in reality a customer could take up the very space they vacate
when they leave service in reality is of course important, however in a
standard analytical model as well as most commercial simulation packages
this would not happen: the model would become deadlocked.
This is why an understanding of this phenomenon from a mathematical point of
view is vital as it helps design modelling frameworks able to easily deal with
deadlock.
For example in the simulation library Ciw, all the theoretic deadlock detection
methods have been implemented (based on the work presented here) and can be used
by the user if they so choose.
We appreciate the reviewer pointing out that this needed clarification and feel
that the numerous amendments made to the paper ensure that this is now clear.

The $-1$ state is simply used as a point of notation, any other symbol could be
used.

\item Reviewer 5 wrote:
\begin{quote}
``It would be great to include in the Introduction a motivating real-world
example related to an open restricted queueing network;''
\end{quote}
This has now been given as its own section, showing the relationship between
community care and secondary care, and is in fact an $\Omega_2$ system
described here.
Further, a article is referenced in which potential deadlock is described as
occurring in a real healthcare scenario, a Swiss hospital.
We now emphasise that deadlock is a phenomenon and problem of the model, not
of reality.
In practice customers are swapped, but current modelling of circular Type I
blocking does not account for this, and will deadlock.
This emphasises the potential application of the work in developing simulation
and analytical models that may be constrained by deadlock despite the system
itself often being able to recover from deadlock.


\item Reviewer 5 wrote:
\begin{quote}
``There are several instances in the paper of using a term and only defining
it later.
For example, the terms ‘open’ and ‘restricted’ at the foot of p.1;''
\end{quote}
This has been addressed when restructuring the introduction.
Definitions for these terms are given at the beginning of the paper.

\item Reviewer 5 wrote:
\begin{quote}
``Definition 1 is far from clear to me.
What is ‘recursive upstream blocking’?
Is the cessation of service referred to in the definition understood to be
permanent?
The term ‘deadlock’ would rather imply so.
Please clarify.
Your reference on p.4 to different types of deadlock suggests to me that
further clarification around this definition would be very helpful.''
\end{quote}
The definition of deadlock has been rewritten for clarity, and to avoid using
undefined terms.
It now reads:
``When there is a subset of blocked customers who are blocked directly or
indirectly by customers in that subset only, then the system is said to
be in deadlock.''
The text also now explains that deadlock implies permanent cessation of
services.

\item Reviewer 5 wrote:
\begin{quote}
``On p.3 l.4 up, what is $B_j$?''
\end{quote}
$B_j$ is the total capacity of each node in the network.
It corresponds to $n_j + c_j$ when using the notation of this paper.
This has been clarified in the text.

\item Reviewer 5 wrote:
\begin{quote}
``The discussion around Figure 2 is far from clear;''
\end{quote}
The discussion on this figure has been rewritten and hopefully this clarifies
the differences between the definitions of deadlocks defined in this paper,
and the deadlocks defined by those authors.

\item Reviewer 5 wrote:
\begin{quote}
``Much of the graph-theoretic terminology used in Section 2 (including
‘weakly’ and ‘strongly’ connected components) will be unfamiliar to many
readers.
What is a ‘wait-for’ graph? These and some other key terms need to be
defined/explained;''
\end{quote}
More clarification of wait-for graphs, and the contribution of this paper,
is given, with new introductory sentences before defining the state digraph.
In addition, a list of graph theoretical terms has been provided before
discussion on the digraph begins.

\item Reviewer 5 wrote:
\begin{quote}
``On p.8 you say that the simulation model described is there to verify the
results of Section 3.
In fact, the latter concern the mean time to deadlock while the simulation
results you report give a much richer description of the distributions of the
time to deadlock (based on order statistics, it appears-thus possibly not
explicitly including the mean);''
\end{quote}
This is correct, much more information of the distribution of the time to
deadlock is available from the simulation results.
This is clarified in the text.
The simulation results and analytical results are used to verify one another.

\item Reviewer 5 wrote:
\begin{quote}
``In your description on p.9 of the Markov chain modelling, you begin by
discussing transition rates but later down the page appear to convert to
transition probabilities (‘the canonical form of an absorbing Markov Chain’).
Since transition rates are the key model descriptors you use throughout,
I suggest you stick to those;''
\end{quote}
Previously there was a confusing duplication of terminology: 'Transition
probabilities' were used to describe both the probability of a customer joining
another node after service, and also the transition probabilities of a Markov
chain.
We have now used `Routing probabilities' to denote the first case, and retained
`Transition probabilities' to denote the second case.


\item Reviewer 5 wrote:
\begin{quote}
``P.12 I believe you need to apply the function $\omega_1$ to both sides of the
inequalities in l.3 of Remark 2;''
\end{quote}
Yes, this has been addressed.

\item Reviewer 5 wrote:
\begin{quote}
``P.12, l.3 up Here (and elsewhere) the superscript of N (the natural numbers)
is used incorrectly;''
\end{quote}
In the description of the state space of the two server system, the following
was changed: $\mathbb{N}^{(n_1+2\times n_2+2)}$ is replaces with
$\mathbb{N}^{(n_1+2)\times (n_2+2)}$.



\end{itemize}

\end{document}