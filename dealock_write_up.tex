\documentclass{article}

\usepackage{fullpage}
\usepackage{setspace}
\usepackage{mathtools}
\usepackage{tikz}
\usepackage{standalone}
\usepackage{float}


\title{Detecting Deadlock in Discrete Event Simulations of Queueing Networks}
\author{Geraint Palmer}
\date{}

\begin{document}
\onehalfspacing

\maketitle

\section{Explanation of Deadlock}

Any open queueing network with feedback loops, at least one service station that has limited queueing capacity, and where individuals can be blocked from joining the queue at the next destination can experience deadlock.
Deadlock occurs when a customer finishes service at node $i$ and is blocked from transitioning to node $j$; however the individuals in node $j$ are all blocked, directly or indirectly, by the blocked individual in node $i$.
That is, deadlock occurs if every individual blocking individual $X$, directly or indirectly, are also blocked.\newline

In figure 1 a simple two node queueing network is shown in a deadlocked state.
Customer $e$ has finished service at node $1$, but remains there as there is not enough queueing space at node $2$ to accept him.
We say he is blocked by customer $i$, as he is waiting for customer $i$ to be released.
Similarly, customer $i$ has finished service at node $2$, but remains there as there is not enough queueing space at node $1$, customer $i$ is blocked by customer $e$.\newline

When there are multiple servers, individuals become blocked by all customers in service or blocked at their destination.
Figure 2 (top) shows two nodes in deadlock, customer $i$ is blocked by both $d$ and $e$, who are both blocked by customer $i$.
However in figure 2 (bottom), customer $i$ is blocked by both $d$ adn $e$, and customer $d$ isn't blocked, and so there is no deadlock.\newline

\begin{figure}[H]
  \includestandalone[width=\textwidth]{images/2nodesindeadlock}
  \caption{Two nodes in deadlock.}
\end{figure}

\begin{figure}[H]
  \begin{tabular}{c c}
    \includestandalone[width=\textwidth]{images/2nodesindeadlockmulti} \\
    \\
    \includestandalone[width=\textwidth]{images/2nodesnotdeadlocked}
  \end{tabular}
  \caption{Two nodes in deadlock (top), and two nodes not deadlocked (bottom).}
\end{figure}



\section{Literature Review}

Most of the literature on blocking conveniently assumes the networks are deadlock-free.
For closed networks of $K$ customers with only one class of customer, \cite{kunduakyildiz89} proves the following condition to ensures no deadlock: for each minimum cycle $C$, $K < \sum_{j\in C} B_j$, the total number of customers cannot exceed the total queueing capacity of each minimum subcycle of the network.
The paper also presents algorithms for finding the minimum queueing space required to ensure deadlock never occurs, for closed cactus networks, where no two cycles have more than one node in common.
This result is extended to multiple classes of customer in \cite{liebeherrakyildiz95}, with more restrictions such as single servers and each class having the same service time distribution.
Here a integer linear program is formulated to find the minimum queueing space assignment that prevents deadlock.
The literature does not discuss deadlock properties in open restricted queueing networks.\newline

General deadlock situations that are not specific to queueing networks are discussed in \cite{coffmanelphick71}.
Conditions for this type of deadlock, also referred to as deadly embraces, to potentially occur are given:
\begin{itemize}
  \item Mutual exclusion: Tasks have exclusive control over resources.
  \item Wait for: Tasks do not release resources while waiting for other resources.
  \item No preemption: Resources cannot be removed until they have been used to completion.
  \item Circular wait: A circular chain of tasks exists, where each task requests a resource from another task in the chain.
\end{itemize}
Dynamic state-graphs are defined, with resources as vertices and requests as edges.
For scenarios where there is only one type of each resource, deadlock arises if and only if the state-graph contains a cylce.\newline

In \cite{choetal95} the vertices and edges of the state graph are given labels in relation to a reference node.
Using these labels \textit{simple bounded circuits} are defined whose existance within the state graph is sufficient to detect deadlock.\newline


\section{Definitions}

\begin{tabular}{ p{5cm} p{10cm} }
  $\left| V(D) \right|$ & The order of the directed graph $D$ is its number of vertices. \\
  Weakly connected component & A weakly connected component of a digraph containing $X$ is the set of all nodes that can be reached from $X$ if we ignore the direction of the edges. \\
  Direct successor & If a directed graph contains an edge from $X_i$ to $X_j$ , then we say that $X_j$ is a direct successor of $X_i$. \\
  Ancestors & If a directed graph contains a path from $X_i$ to $X_j$ , then we say that $X_i$ is an ancestor of $X_j$. \\
  Decendents & If a directed graph contains a path from $X_i$ to $X_j$ , then we say that $X_j$ is a descendant of $X_i$. \\
  $\text{deg}^{\text{out}}(X)$ & The out-degree of $X$ is the number of outgoing edges emanating from that vertex. \\
  Subgraph & A subgraph $H$ of a graph $G$ is a graph whose vertices are a subset of the vertex set of $G$, and whose edges are a subset of the edge set of $G$. \\
\end{tabular}
\newline
NEED CONSISTANT NOTATION, REFERENCES FOR DEFINITIONS.


\section{Explaining Digraph}

Presented is a method of detecting when deadlock occurs in an open queueing network $Q$ with $N$ nodes, using a dynamic directed graph, the state graph.
Let the number of servers in node $i$ be denoted by $c_i$.\newline

Define $D(t)$ as the state graph of $Q$ at time $t$.
Each vertex of $D(t)$ is an occupied server, occupied by an individual who is either in service or blocked, and so $\left| V\left(D\left(t\right)\right) \right| \leq \sum_{i=1}^N c_i, \quad \forall t \geq 0$.
The state graph shows the blockage relationships between servers: a directed edge from one $X_a$ to $X_b$ represents that the customer occupying $X_a$ has finished service and is waiting for the $X_b$ to release its occupant before he can be realsed himself.\newline

The state graph $D(t)$ can be partitioned into $N$ service-station subgraphs, $d_1(t), d_2(t), ..., d_N(t)$, where the vertices of $d_i(t)$ represent the servers of node $i$.
The vertex set of each subgraph is static over time, however their edge sets may change.\newline

The state graph is dynamically built up as follows.
When an individual finishes service at node $i$, and this individuals next destination is node $j$, but there is not enough queueing capacity for $j$ to accept that individual, then that individual remains at node $i$ and becomes blocked.
At this point $c_j$ directed edges between this individual's server and the vertices of $d_j(t)$ are created in $D(t)$.\newline

When an individual is released and another customer occupies their server, that servers out-edges are removed.
When an individual is released and there isn't another customer to occupy that server, then all edges incedent to that server are removed.\newline

NEED LOTS OF PARALLEL DIAGRAMS, REALITY vs STATE GRAPH.



\section{Theorem}

\subsection*{Theorem}

A deadlock situation arises at time $t$ if and only if there exists a weakly connected component of $D(t)$ containing no vertices with $deg^{out} = 0$.\newline

\subsection*{Observations}

Consider one weakly connected component $G(t)$ of $D(t)$.
Consider the node $X_a \in G(t)$. If $X_a$ is unoccupied, then $X_a$ has no incident edges.
Consider the case when $X_a$ is occupied by individual $a$, whose next destination is node $j$.
Then $X_a$'s direct successors are the  servers occupied by individuals who are blocked or in service at node $j$.
We can interpret all $X_a$'s decendents as the servers whose occupants are directly or indirectly blocking $a$, and we can interpret all $X_a$'s ancestors as those servers whose individuals who are being blocked directly or indirectly by $a$.\newline

Note that the only possibilities for $\text{deg}^{\text{out}}(X_a)$ are being 0 or $c_j$.
If $\text{deg}^{\text{out}}(X_a) = c_j$ then $a$ is blocked by all its direct successors.
The only other situation is that $a$ is not blocked, and $X_a \in G(t)$ because $a$ is in service at $X_a$ and blocking other individuals, in which case $\text{deg}^{\text{out}}(X_a) = 0$.\newline

It is clear that if any of $X_a$'s decendents are occupied by individuals who are not blocked, then we do not have deadlock at time $t$; and if all of $X_a$'s descendents are occupied by blocked individuals, then the system is deadlocked at time $t$.
We also know that by definition all of $X_a$'s ancestors are occupied by blocked individuals.\newline

Also note that if a service-station subgraph $d_i(t)$ contains edges, then there is an individual $X_a$ in that service station is being blocked by himself.
This can only happens if there is another individual being blocked by $X_a$, and that individual became blocked before $X_a$ became blocked.
It is also true that every vertex in $d_i(t)$ has the same ancestors.\newline

\subsection*{Proof}

Consider one weakly connected component $G(t)$ of $D(t)$ at time $t$.
All vertices of $G(t)$ are either decendents of another vertex and so are occupied by an individual who is blocking someone; or are ancestors of another vertex, and so are occupied by someone who is blocked.\newline

Assume that $G(t)$ contains a vertex $X$ such that $\text{deg}^{\text{out}}(X) = 0$.
This imples that $X$'s occupant is not blocked, so $X$ must be a descendent of another vertrex.
Therefore $Q$ is not deadlocked as there exists a vertex whose descendents are not all blocked.\newline

Now assume that we have deadlock.
For a given vertex $X$, all decendents of $X$ are are occupied by individuals who are blocked, and so must have out-degrees greater than 0.
However, as our choice of $X$ was arbitrary, this must be true for all members of $G$, and so no members of $G(t)$ have out-degree of 0.\newline

\bibliographystyle{plain}
\bibliography{refs}





\end{document}
